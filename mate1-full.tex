\documentclass[11pt,titlepage]{article}
\usepackage[a4paper,
            left=3cm,
            right=3cm,
            top=2cm,
            bottom=2cm]{geometry}
\usepackage[spanish]{babel}
\usepackage[none]{hyphenat}
\usepackage[utf8]{inputenc}
\usepackage[titletoc]{appendix}
\usepackage{csquotes}

\usepackage{etoolbox}
\newbool{DEBUG}
%\booltrue{DEBUG}
\boolfalse{DEBUG}

\usepackage{amsmath,mathtools,cancel,multicol}
\usepackage{graphicx,booktabs,tabularx,amssymb,amsfonts}
\graphicspath{{./media/}}
\usepackage[shortlabels]{enumitem}
\usepackage{pdfpages}

\usepackage{verbatim}

\usepackage{color,soul}
%\sethlcolor{orange}

\usepackage{tcolorbox}
\newtcolorbox{commBoxy}{
	colback=red!5!white,
	colframe=red!75!black}
\newtcolorbox{titleBoxy}[1]{
	colback=red!5!white,
	colframe=red!75!black,
	fonttitle=\mdseries,
	title=#1}

\usepackage[pdfauthor={Alejandro Nahuel Heir},
            colorlinks,
			linkcolor=red]{hyperref}

%Custom Commands:
\newcommand{\commLim}[2]{\lim_{#1 \to #2}}
\newcommand{\displayLim}[2]{\displaystyle \commLim{#1}{#2}}

\newcommand{\littleTitle}[1]{
	\noindent \ignorespaces
	\small \textbf{#1} \normalsize
	\ignorespaces \ignorespacesafterend
}

\newcommand{\comillas}[1]{``#1''}

\DeclareMathOperator{\arcsinh}{arcsinh}
\DeclareMathOperator{\arccosh}{arccosh}
\DeclareMathOperator{\arctanh}{arctanh}

\begin{document}

\begin{titlepage}
	\centering
	{\Large Instituto Tecnológico Buenos Aires \par}
	\vspace{2cm}
	{\huge Matemática I - 93.17 \par}
	\vspace{2cm}
	{\Huge Resumen Práctico \par}
	\vspace{2cm}
	{\ttfamily Alejandro Nahuel Heir \par}
	\vspace{0.5cm}
	{\ttfamily 2021 \par}
\end{titlepage}

\renewcommand{\abstractname}{\LARGE Disclaimer}
\begin{abstract}
	Usar el presente material a modo de refuerzo y/o repaso de los contenidos, no como alternativa al material de la cátedra.\\
	No contiene ninguna justificación teórica.\\
	Fue realizado principalmente a modo de práctica para con \LaTeX.\\
	Al final del documento se encuentran dos PDF \textbf{de terceros} sobre trigonometría, útiles para las identidades.\\
	Cualquier sugerencia, correción o similar sobre los contenidos y/o formato del documento, es bienvenida: %
	\href{mailto:aheir@itba.edu.ar}{aheir@itba.edu.ar}.\\
	Las actualizaciones del documento pueden encontrarse en su %
	\href{https://github.com/anheir/mate1-full}{repositorio} en GitHub
	
\end{abstract}

\tableofcontents
\listoffigures
\listoftables
%\setcounter{tocdepth}{4}
%\setcounter{secnumdepth}{4}
%\setcounter{tocdepth}{5}
%\setcounter{secnumdepth}{5}
\newpage

\part{Primer Parcial}
\section{Límites}

\subsection{Cambio de variable}
Sean $\lim_{y \to b} g(y)=L, \lim_{x \to a} f(x)=b, f(x) \neq b \text{ en un entorno reducido de } a$,\\
entonces
\begin{equation}\label{cambiovar}
	\lim_{x \to a} g(f(x)) = \lim_{y \to b} g(y) = L
\end{equation}

\subsection{Cero por acotada}
Sean $\lim_{x \to a} g(x) = 0, \nexists \lim_{x \to a} f(x) \text{, con \textit{f} acotada en un entorno reducido de \textit{a}}$,\\
entonces
\begin{equation}\label{ceroacotada}
		\lim_{x \to a} \overbrace{f(x)}^{acotada} \cancelto{0}{g(x)} = 0
\end{equation}

\subsection{Lema del Sandwich}
Sean $f(x) \leq g(x) \leq h(x), \forall x \in E^{\ast}_{(a,r)} \text{con } r>0.$
Si $\lim_{x \to a} f(x) = \lim_{x \to a} h(x) = L$, entonces
\begin{equation}\label{sandwich}
	\lim_{x \to a} g(x) = L
\end{equation}

\littleTitle{Observación}\par
Sea $\lim_{x \to a} |f(x)| = 0$, y sabiendo que $-|f(x)| \leq f(x) \leq |f(x)|$, se deduce por Lema del Sandwich que
\begin{equation}
	\lim_{x \to a} |f(x)| = 0 \Rightarrow \lim_{x \to a} f(x) = 0
\end{equation}

\subsection{Sobre límites laterales}
\begin{equation*}
	\nexists \lim_{x \to a} f(x) \text{ si }
	\begin{cases}
		\lim_{x \to a^{+}} f(x) = L_1 \\
		\lim_{x \to a^{-}} f(x) = L_2
	\end{cases}
	L_1 \neq L_2
\end{equation*}
\begin{equation*}
	\nexists \lim_{x \to a} f(x) \text{ si }
	\begin{cases}
		\commLim{x}{a^{+}} f(x) = L \\
		\nexists \commLim{x}{a^{-}} f(x)
	\end{cases}
\end{equation*}
\vspace{0.5cm}
\begin{equation}
	\therefore \commLim{x}{a} f(x) = L \Leftrightarrow \commLim{x}{a^{+}} f(x) = \commLim{x}{a^{-}} f(x) = L
\end{equation}

\subsection{Límites importantes}
\subsubsection{Trigonométricos}
\begin{multicols}{3}
	\begin{enumerate}[label=\alph*.]
		\item $ \displayLim{x}{0} \dfrac{\sin x}{x} = 1 $
		\item $ \displayLim{x}{0} \frac{x}{\sin x} = 1 $
		      %\item $ \displayLim{x}{0} \frac{\cos x}{x} = \infty $
		\item $ \displayLim{x}{0} \frac{\tan x}{x} = 1 $
		\item $ \displayLim{x}{0} \frac{x}{\tan x} = 1 $
		\item $ \displayLim{x}{0} \frac{\arctan x}{x} = 1 $
	\end{enumerate}
\end{multicols}

\subsubsection{Relativos a \textit{e}}
\littleTitle{Igualdad importante}\par
	\begin{equation}
		\boldsymbol{f(x)^{g(x)} = e^{g(x) \ln(f(x))}}
	\end{equation}

\begin{enumerate}[label=\alph*.]
	\item $ \displayLim{x}{a} \left(1+\frac{1}{f(x)}\right)^{f(x)} = e, \quad	\text{si} \commLim{x}{a} f(x) = \infty $
	\item $ \displayLim{x}{a} (1+f(x))^{\frac{1}{f(x)}} = e, \quad	\text{si} \commLim{x}{a} f(x) = 0 $
	\item $ \displayLim{x}{a} \frac{\ln(1 + f(x))}{f(x)} = 1, \quad	\text{si} \commLim{x}{a} f(x) = 0 $
	\item $ \displayLim{x}{a} \frac{e^{f(x)}-1}{f(x)} = 1, \quad \text{si} \commLim{x}{a} f(x) = 0 $
\end{enumerate}

\subsection{Sobre límites infinitos}
\begin{multicols}{2}
	\begin{enumerate}[label=\alph*.]
		\item $ \displayLim{x}{0} \frac{k}{x} = \infty$, \quad con $ k \in \mathbb{R} - \left\{0 \right\} $
		\item $ \displayLim{x}{\infty} \frac{k}{x} = 0$, \quad con $ k \in \mathbb{R} $
		\item $ \displayLim{x}{\infty} kx = \infty$, \quad con $k \in \mathbb{R} - \left\{0\right\} $
		\item $ \displayLim{x}{\infty} k + x = \infty$, \quad con $ k \in \mathbb{R} $
	\end{enumerate}
\end{multicols}


\vspace{1cm}
\section{Continuidad}
\subsection{Definición}
$f$ es continua en $a \in \mathbb{R}$ si:
\begin{itemize}
	\item $a \in Dom(f)$
	\item $\exists \commLim{x}{a} f(x)$
	\item $\commLim{x}{a} f(x) = f(a)$
\end{itemize}

\subsection{Propiedades}
Sean $f$ y $g$ continuas en $a \in \mathbb{R}$:
\begin{multicols}{2}
	\begin{itemize}
		\item $cf$ es continua en $a$, $\forall c \in \mathbb{R}$
		\item $f \pm g$ es continua en $a$
		\item $fg$ es continua en a
		\item $\dfrac{f}{g}$ es continua en $a$, si $g(a) \neq 0$
	\end{itemize}
\end{multicols}
	
\subsection{Funciones continuas en todo su dominio}
\begin{multicols}{2}
	\begin{itemize}
		\item Polinómicas
		\item Trigonométricas (directas o inversas)
		\item Exponenciales
		\item Logarítmicas
		\item Raíces (excepto $\sqrt[n]{x}$ en $x = 0$ para $n$ pares)
	\end{itemize}
\end{multicols}

\subsection{Composición}
Si $f$ es continua en $x=a$, y $g(z)$ es continua en $z=f(a)$, entonces $h(x) = (g\circ f)(x)$ es continua en $a$.

\subsection{Discontinuidades}
$f$ es discontinua en $a \in \mathbb{R}$ si se cumple \underline{\emph{al menos una}} de las siguientes condiciones:
\begin{itemize}
	\item $a \notin Dom(f)$
	\item $\nexists \commLim{x}{a} f(x)$ 
	\item $\commLim{x}{a} f(x) \neq f(a)$ 
\end{itemize}

\subsubsection{Evitables}
$a \in \mathbb{R}$ es discontinuidad evitable si se cumple simultáneamente
\begin{multicols}{3}
	\begin{itemize}
		\item $f(a) = L_1 \in \mathbb{R}$
		\item $\exists \displayLim{x}{a} f(x) = L_2 \in \mathbb{R}$
		\item $L_1 \neq L_2 \neq \infty$
	\end{itemize}
\end{multicols}
\emph{Si se redefine $f(x)$ en $x = a$ como $f(a) = L_2$, $f$ pasa a ser continua en $a$.}

\subsubsection{No evitables o esenciales}

\littleTitle{Tipo salto}\par
$a \in \mathbb{R}$ es discontinuidad esencial tipo salto si
\begin{equation*}
	\commLim{x}{a^{-}} f(x) = L_1 \neq \commLim{x}{a^{+}} f(x) = L_2, \quad L_1, L_2 \in \mathbb{R}
\end{equation*}

\littleTitle{Tipo asíntota (vertical)}\par
$a \in \mathbb{R}$ es discontinuidad esencial tipo asíntota si se cumple \underline{\emph{alguna}} de las siguientes igualdades
\begin{equation*}
	\commLim{x}{a^{-}} f(x) = \infty \qquad \text{o} \qquad \commLim{x}{a^{+}} f(x) = \infty
\end{equation*}

\littleTitle{\comillas{De otro tipo}}\par
$a \in \mathbb{R}$ es discontinuidad esencial de otro tipo si no es ninguna de las anteriores. Por ejemplo, $a = 0, f(a), \text{ con } f(x) = \sin \dfrac{1}{x}$
\begin{figure}[htb!]
	\centering 
	\includegraphics[scale=1.5]{discontinuidad_otro_tipo.png}
	\caption{Gráfico de la función $\sin \dfrac{1}{x}$}
	\label{fig:discontinuidad_otro_tipo}
\end{figure}

\pagebreak

\subsection{Teorema de Bolzano - T.B.} \label{uso_bolzano}
\begin{commBoxy}
	Si $f$ es continua en $[a,b]$, y $f(a)f(b) < 0$ (\emph{tienen signos opuestos}), entonces 
	\begin{equation}
		\boldsymbol{\exists c \in \left(a,b\right) / f(c) = 0} \text{ (\emph{al menos una raiz})}
	\end{equation}
\end{commBoxy}
Si $f$ es continua en un intervalo abierto $\left(a,b\right)$, se debe cumplir que $\commLim{x}{a^{+}} f(x) = f(a)$, y que $\commLim{x}{b^{-}} f(x) = f(b).$

\subsubsection{Uso} \label{usobolzano}
\littleTitle{Hallar raíces mínimas de una función}\par
Dada una $f(x)$ igualada a $0$, continua en un intervalo dado, hallar por tanteo dos valores de $x$ (que pertenezcan 
al intervalo donde $f$ es continua) para los cuales $f(x)$ tenga \emph{distinto signo}. Luego, por \emph{T.B.}, esa 
función tendrá \emph{al menos} una raíz entre esos dos valores de $x$ elegidos.\par
Cabe resaltar que esto puede emplearse también para conocer soluciones mínimas de una ecuación igualada a $0$.

\subsection{Corolario del Teorema de Bolzano - C.T.B.} \label{ctb}
\begin{commBoxy}
	Sea $f$ continua en $(a,b)$, con $f(x) \neq 0 \ \forall x \in (a,b)$, entonces $f$ mantiene su signo en $(a,b)$. Es decir:
	\begin{equation}
		\boldsymbol{f(x) > 0 \ \forall x \in (a,b) \qquad \text{ó} \qquad f(x) < 0 \ \forall x \in (a,b)} 
	\end{equation}
\end{commBoxy}

\subsubsection{Uso}
\littleTitle{Hallar conjuntos de positividad y negatividad de una función}\par
Dada una función, y conociendo su dominio y conjunto de ceros, se puede \comillas{partir} el dominio de la función en intervalos donde la misma es
continua y no nula (esto último sabiendo el conjunto de ceros). A lo largo de cada intervalo, si la función es continua en él, mantendrá 
su signo; basta tomar una $x$ cualquiera en ese intervalo y evaluarla en la función para saber el signo en todo ese intervalo.

\subsection{Teorema del Valor Intermedio - T.V.I.}
Sea $f$ continua en $[a,b]$, $f(a) = c, \ f(b) = d, \ c \neq d,$ entonces:
\begin{enumerate}[label=\alph*.]
	\item $c < d \Rightarrow [c,d] \subset f\left([a,b]\right)$
	\item $d < c \Rightarrow [d,c] \subset f\left([a,b]\right)$
\end{enumerate}

\littleTitle{Generalización}\par
Sea $f$ continua en (a,b), $\commLim{x}{a^{+}} f(x) = c, \ \commLim{x}{b^{-}} f(x) = d$, entonces:
\begin{enumerate}[label=\alph*.]
	\item $c < d \Rightarrow [c,d] \subset f\left([a,b]\right)$
	\item $d < c \Rightarrow [d,c] \subset f\left([a,b]\right)$
\end{enumerate}

\pagebreak

\subsection{Teorema para evaluar sobreyectividad}
Sea $f$ continua en $(a,b)$, donde $a$ podría ser $-\infty$ y $b$ $+\infty$. Si
\begin{equation*}
	\commLim{x}{a^{+}} f(x) = -\infty \quad \wedge \quad \displayLim{x}{b^{-}} f(x) = +\infty \\
\end{equation*}
\begin{center}
\Large{\emph{o}}
\end{center}
\begin{equation*}
	\commLim{x}{a^{+}} f(x) = +\infty \quad \wedge \quad \displayLim{x}{b^{-}} f(x) = -\infty \\
\end{equation*}
entonces
\begin{equation*}
	f\left((a,b)\right) = \mathbb{R}, \ \therefore \ f \ \text{\emph{es sobreyectiva}}
\end{equation*}


\vspace{1cm}
\section{Derivadas}

\subsection{Ecuaciones de rectas}
Sea $x_0 \in \mathbb{R}$ el punto en el cual la recta es tangente o normal a la función $f$.
\subsubsection{Recta tangente}
\begin{equation}
	r_{T}(x) = f(x_0) + f'(x_0)(x - x_0)
\end{equation}
\subsubsection{Recta normal}
\begin{equation}
	r_{N}(x) = f(x_0) + \left(\frac{-1}{f'(x_0)}\right)(x - x_0)
\end{equation}

\subsection{Derivada por definición}
La derivada de $f$ en $x_0$ es, si existe, el valor del límite del siguiente cociente incremental:
\begin{equation}
	\begin{aligned}
		f'(x_0) = \commLim{x}{x_0} \frac{\Delta f}{\Delta x} &= \commLim{x}{x_0} \frac{f(x) - f(x_0)}{x - x_0} \\ \\
		&= \commLim{h}{0} \frac{f(x_{0} + h) - f(x_0)}{h}
	\end{aligned}
\end{equation}

\subsection{Teorema en relación a la continuidad}
\begin{itemize}
	\item Si $\exists f'(x_0) \Rightarrow f$ es continua en $x_0$
	\item Si $f$ es discontinua en $x_0 \Rightarrow \nexists f'(x_0)$
\end{itemize}
\emph{$f$ continua en $x_0$ NO necesariamente implica $\exists f'(x_0)$.} Por ejemplo:
\begin{gather*}
	f(x) = |x| \text{ es continua en } \mathbb{R} \\ \\
	\commLim{x}{0} \frac{|x| - 0}{x} = \commLim{x}{0} sg(x), \quad \text{pero} \quad \nexists \commLim{x}{0} sg(x) \\ \\
	\therefore |x| \ \text{NO es derivable en} \ x_0 = 0
\end{gather*}

\subsection{Reglas de derivación}
Sean $f$ y $g$ derivables en $x_0 \in \mathbb{R}$:
\begin{itemize}
	\item[1.] $(cf)'(x_0) = cf'(x_0), \quad \forall c \in \mathbb{R}$
	\item[2.] $(f \pm g)'(x_0) = f'(x_0) \pm g'(x_0)$ 
	\item[3.] $(fg)' = f'(x_0)g(x_0) + f(x_0)g'(x_0)$
	\item[4.] $\left(\dfrac{f}{g}\right)' = \dfrac{f'(x_0)g(x_0) - f(x_0)g'(x_0)}{g^{2}(x_0)}, \quad g(x_0) \neq 0$
\end{itemize}

\subsection{Regla de la cadena}
Sea $f(x)$ derivable en $x_0$, y $g(y)$ derivable en $y_0$, entonces $h(x) = (g \circ f)(x)$ es derivable en $x_0$, y:
\begin{equation}
	h'(x_0) = (g \circ f)'(x) = g'\left(f(x_0)\right) \cdot f'(x_0)
\end{equation}

\subsection{Derivadas notables}
\begin{table}[h!]
	\centering
	\setlength{\tabcolsep}{10pt} % Default value: 6pt
	\renewcommand{\arraystretch}{2.5} % Default value: 1
	\begin{tabularx}{\textwidth}{lX||lX}
		\toprule
		$f(x)$            & $f'(x)$                                  & $f(x)$       & $f'(x)$                                          \\
		\midrule
		$c\in \mathbb{R}$ & 0                                        & $\cosh x$    & $\sinh x$                                        \\
		$x$               & 1                                        & $\sinh x$    & $\cosh x$                                        \\
		$a^{x}$           & $\ln(a)a^{x}, \quad a>0$                 & $\tanh x$    & $\dfrac{1}{\cosh^2 x}$                           \\
		$\ln x$           & $\dfrac{1}{x}$                           & $\arcsin x$  & $\dfrac{1}{\sqrt{1-x^2}}, \quad x\in (-1,1)$     \\
		$\log_{a} x$      & $\dfrac{1}{x\ln a}$                      & $\arccos x$  & $\dfrac{-1}{\sqrt{1-x^2}}, \quad x\in (-1,1)$    \\
		$|x|$             & $sg(x), \quad x \neq 0$                  & $\arctan x$  & $\dfrac{1}{1+x^2}, \quad x\in \mathbb{R}$        \\
		$x^{n}$           & $nx^{n-1}, \quad x>0, \ n\in \mathbb{R}$ & $\arcsinh x$ & $\dfrac{1}{\sqrt{1+x^2}}, \quad x\in \mathbb{R}$ \\
		$\sin x$          & $\cos x$                                 & $\arccosh x$ & $\dfrac{1}{\sqrt{x^2 -1}}, \quad x>1$            \\
		$\cos x$          & $\sin x$                                 & $\arctanh x$ & $\dfrac{1}{x^2 -1}, \quad x\in (-1,1)$           \\
		$\tan x$          & $\dfrac{1}{\cos ^{2}x}$                  &              &                                                  \\
		\bottomrule
	\end{tabularx}
	\caption{Derivadas notables}
\end{table}

\subsection{Sobre funciones partidas}
Siempre se debe analizar por definición la continuidad y derivabilidad en los valores de $x$ donde la función se parte.

\subsection{Teorema de la Función Inversa}
Sea $f: \ (a,b) \to \mathbb{R}$ continua e \emph{inyectiva}, entonces:
\begin{itemize}
	\item[1°] $Im(f)$ es un intervalo $(c,d)$, y $\exists f^{-1}: (c,d) \to (a,b)$, la cual es continua.
	\item[2°] Si $f$ es derivable en $x_0 \in (a,b)$, y $f'(x_0) \neq 0$, entonces $f'$ es derivable en $y_0 = f(x_0)$, y  
\end{itemize}
\begin{equation}
	\left(f^{-1}\right)'(y_0) = \frac{1}{f'(x_0)} = \frac{1}{f'\left(f^{-1}(y_0)\right)}
\end{equation}

\subsection{Aproximación lineal de un función}
\begin{equation}
	\begin{aligned}
		L(x) &= f(x_0) + f'(x_0)(x - x_0), \qquad \text{aproximación lineal de $f$ en $x_0$} \\ \\
		f(x) &\simeq L(x) \text{ cuando } x - x_0 \ll 1
	\end{aligned}
\end{equation}

\subsection{Aproximación diferencial de una función}
\begin{equation}
	\begin{aligned}
		df &= f'(x_0) \cdot \Delta x, \qquad \text{diferencial de $f$ en $x_0$} \\ \\
		\Delta f &\simeq df \text{ cuando } \Delta x \ll 1
	\end{aligned}
\end{equation}

\vspace{1.5cm}
\section{Teorema del Valor Medio}

\subsection{Teorema del Valor Intermedio - T.V.I.}
Sea $f$ continua en $[a,b]$, $f(a) = c$, $f(b) = d$, $c \neq d$:
\begin{itemize}
	\item[a.] $c < d \Rightarrow [c,d] \subset f\left([a,b]\right)$
	\item[b.] $d < c \Rightarrow [d,c] \subset f\left([a,b]\right)$
\end{itemize}
\littleTitle{Generalización}\par
Sea $f$ continua en (a,b), $\commLim{x}{a^{+}} f(x) = c \neq \commLim{x}{b^{-}} f(x) = d$, entonces:
\begin{itemize}
	\item[a.] $c < d \Rightarrow [c,d] \subset f\left([a,b]\right)$
	\item[b.] $d < c \Rightarrow [d,c] \subset f\left([a,b]\right)$
\end{itemize}

\subsection{Teorema de Fermat}
Sea $f: (a,b) \to \mathbb{R} / f$ alcanza un extremo en un $x_0 \in (a,b)$, entonces
\begin{gather}
	\nexists f'(x_0) \qquad \text{ó} \qquad f'(x_0) = 0
\end{gather}

\subsection{Teorema de Weierstrass - T.W.} \label{tw}
\begin{commBoxy}
	Sea $f: [a,b] \to \mathbb{R}$ continua, \emph{entonces} $f$ alcanza un $M$ (máximo) y $m$ (mínimo) en $[a,b]$.
	Además, por \emph{T.V.I.}:
	\begin{gather*}
		Im(f) = [m,M] \\ \\
		m \leq f(x) \leq M, \ \forall x \in [a,b] \\ \\
		\text{Si } m = M \Rightarrow f(x) \text{ es \emph{constante} en } [a,b]
	\end{gather*}
\end{commBoxy}

\subsection{Teorema de Rolle - T.R.}
\begin{commBoxy}
	\littleTitle{Hipótesis}
	\begin{itemize}
		\item $f: [a,b] \to \mathbb{R}$ continua; derivable en $(a,b)$.
		\item $f(a) = f(b)$
	\end{itemize}
	
	\littleTitle{Tesis}
	\begin{gather*}
		\exists c \in (a,b) / f'(c) = 0, \qquad \text{\emph{se afirma al menos unas raiz de }$f'$}
	\end{gather*}
\end{commBoxy}

\subsection{Corolario del Teorema de Rolle - C.T.R.} \label{ctr}
\begin{commBoxy}
	\littleTitle{Hipótesis}
	\begin{itemize}
		\item $f$ continua en $e{[a,b]}_{(a,b)}$; derivable en $(a,b)$.
		\item $f'(x) = 0$ tiene exactamente $k$ soluciones en $(a,b)$
	\end{itemize}

	\littleTitle{Tesis}\par
	\begin{center}
		$f(x) = 0$ tiene \emph{como máximo} $k + 1$ soluciones en ${[a,b]}_{(a,b)}$
	\end{center}
\end{commBoxy}

\subsection{Aplicaciones del C.T.R.}
\subsubsection{Afirmar inyectividad de una función}
Sea $f$ continua en ${[a,b]}_{(a,b)}$, derivable en $(a,b)$, $f'(x) \neq 0 \ \forall x \in (a,b)$, entonces
\begin{center}
	\vspace{0.25cm}
	$f$ es inyectiva en ${[a,b]}_{(a,b)}$ \\
	\vspace{0.25cm}
	\emph{"Si la derivada no se anula $\Rightarrow f$ es inyectiva"}
\end{center}

\subsubsection{Determinar raíces de una función}
Consiste en tomar una función igualada a $0$, buscar sus soluciones máximas con \hyperref[ctr]{\emph{C.T.R.}}, y luego 
las soluciones mínimas con \hyperref[uso_bolzano]{\emph{T.B.}}. Si el número de ambas soluciones coinicide (\emph{debería}), 
\begin{center}
	$\therefore$ se tiene la cantidad exacta de soluciones.
\end{center} \par
Es posible que para hallar las soluciones de $f'(x) = 0$ haya que aplicarle \emph{T.B.} a $f'$ y \emph{C.T.R.} 
a $f''$, y concluir en el número de soluciones de $f'(x) = 0$, para luego, recién, tener soluciones máximas de 
$f(x) = 0$.

\subsection{Teorema del Valor Medio de Lagrange - T.V.M.L.} \label{tvml}
\begin{commBoxy}
	\littleTitle{Hipótesis}
	\begin{itemize}
		\item $f$ continua en $[a,b]$, derivable en $(a,b)$.
	\end{itemize}

	\littleTitle{Tesis}
	\begin{gather*}
		\underbrace{\exists c \in (a,b)}_{\text{al menos uno}} / \frac{f(b) - f(a)}{b - a} = f'(c) 
	\end{gather*}
\end{commBoxy}

\subsection{Corolarios del T.V.M.L.}
\subsubsection{Corolario I}
Sea $f$ continua en ${[a,b]}_{(a,b)}$, $f'(x) = 0 \ \forall x \in (a,b)$
\begin{gather*}
	\Rightarrow \ f \ \text{es constante} \quad (f(x) = k \in \mathbb{R}) \quad \forall x \in {[a,b]}_{(a,b)}
\end{gather*}

\subsubsection{Corolario II}
Sean $f$ y $g$ derivables en $(a,b)$, $f'(x) = g'(x) \ \forall x \in (a,b)$
\begin{gather*}
	\Rightarrow \ \exists k \in \mathbb{R} / g(x) = f(x) + k, \ \forall x \in (a,b)
\end{gather*}

\subsubsection{Corolario III \footnotesize{(para desiguldades entre funciones)}}
\begin{commBoxy}
	\littleTitle{Hipótesis}
	\begin{itemize}
		\item $f$ y $g$ continuas en ${[a,b]}_{[a,b)}$, derivables en $(a,b)$.
		\item $f(a) \leq g(a)$
		\item $f'(x) < g'(x), \ \forall x \in (a,b)$
	\end{itemize}

	\littleTitle{Tesis}
	\begin{gather*}
		f(x) < g(x), \ \forall x \in {(a,b]}_{(a,b)}
	\end{gather*}
\end{commBoxy}

\subsubsection{T.V.M.L. para desigualdades}
Tomar un $f(x)$, aplicarle \hyperref[tvml]{\emph{T.V.M.L.}} \comillas{entre a y b}, acotar la $f'(c)$ (ya que $c \in (a,b)$, (o 
usar lo que convenga)).\par
\comillas{Desarrollar}; evaluar $f'(x)$ en $c$ y usar las acotaciones; \comillas{acotando $f'(c)$ se prueban 
desigualdades}.

\subsection{Teorema del Valor Medio de Cauchy}
\begin{commBoxy}
	\littleTitle{Hipótesis}
	\begin{itemize}
		\item $f$ y $g$ continuas en $[a,b]$, derivables en $(a,b)$.
		\item $g'(x) \neq 0 \ \forall x \in (a,b)$ \footnotesize(implicando que $g$ es inyectiva)
	\end{itemize}

	\littleTitle{Tesis}
	\begin{gather*}
		\exists c \in (a,b) / \frac{f(b) - f(a)}{g(b) - g(a)} = \frac{f'(c)}{g'(c)} \\ 
		\text{\footnotesize (generalización del \emph{T.V.M.L.})}
	\end{gather*}
\end{commBoxy}

\subsection{Regla de L'Hospital}
Sean $f$ y $g$ derivables en $E^{*}_{(a,r)} = (a - r, a + r) - \{a\}$, y
\begin{align*}
	\text{1°}& \ \ \ \ \commLim{x}{a} f(x) = \commLim{x}{a} g(x) = 0 &\text{o} \qquad &\commLim{x}{a} f(x) = 
	\commLim{x}{a} g(x) = \infty \\
	\text{2°}& \ \ \ \ g'(x) \neq 0 \ \forall x \in E^{*}_{(a,r)} &\text{y} \qquad &\commLim{x}{a} \frac{f'(x)}{g'(x)} = L, \ L \in
	\mathbb{R} \lor L = \infty
\end{align*}
\begin{gather*}
	\Rightarrow \commLim{x}{a} \frac{f(x)}{g(x)} = \commLim{x}{a} \frac{f'(x)}{g'(x)} = L
\end{gather*}


\newpage
\part{Segundo Parcial}

\section{Aplicaciones de la Derivada}
\subsection{Asíntotas}
\subsubsection{Horizontales}
Son de la forma $y = k \in \mathbb{R}$; tomar $\displayLim{x}{\infty} f(x)$

\subsubsection{Verticales}
Son de la forma $x = k \in \mathbb{R}$; tomar $\displayLim{x}{x_0} f(x)$, con $x_0 \notin Dom(f)$

\subsubsection{Oblicuas}
Son de la forma $y = mx + b$.
\begin{itemize}
	\item Si $\exists m \ \Rightarrow \ m = \displayLim{x}{\infty} \frac{f(x)}{x} \qquad (m \in \mathbb{R}, \ m \neq \pm \infty)$
	\item Si $\exists b \ \Rightarrow \ b = \displayLim{x}{\infty} (f(x) - mx) \qquad (b \in \mathbb{R})$
\end{itemize}

\subsection{Monotonía \footnotesize (implica inyectividad)}
\begin{center}
	$f$ es monótona en $I$ $\Leftrightarrow$ es estríctamente creciente o decreciente en $I$
\end{center} 
\littleTitle{Teorema} \par
Sea $f$ continua en $I = {[a,b]}_{(a,b)}$ y derivable en $(a,b)$
\begin{enumerate}[label=\alph*.]
	\item Si $f'(x) > 0 \quad \forall x \in (a,b) \ \Rightarrow \ f$ es \emph{estríctamente creciente} en $I = %
	{[a,b]}_{(a,b)}$
	\item Si $f'(x) < 0 \quad \forall x \in (a,b) \ \Rightarrow \ f$ es \emph{estríctamente decreciente} en $I = %
	{[a,b]}_{(a,b)}$
\end{enumerate} 
\begin{center}
	\footnotesize (de ser necesario, usar \hyperref[ctb]{\emph{C.T.B.}} para asegurar signo de $f'$ en diferentes intervalos)
\end{center}

\subsection{Extremos locales o relativos}
$f$ alcanza un $M$ (máximo) o $m$ (mínimo) local en $x_0$ si
\begin{gather*}
	\exists \delta > 0 \ / \ f(x_0) \ \underbrace{\geq_{\scriptscriptstyle(M)}}_{\leq_{\scriptscriptstyle(m)}} f(x) \quad \forall x \in (x_0 - \delta, x_0 + \delta)
\end{gather*} \par
Un extremo local es un $M$ o $m$ local.

\subsection{Puntos críticos de una función - \emph{PC}}
$x_0$ es \emph{PC} de $f$ si $x_0 \in Dom(f)$ y
\begin{gather*}
	\text{a)} \ \ \nexists f'(x_0) \qquad \veebar \qquad b) \ \ f'(x_0) = 0
\end{gather*} \par
\begin{center}
	Si $x_0$ es un extremo local $\Rightarrow$ $x_0$ es \emph{PC}
\end{center}

\subsection{Criterio de la 1ra derivada para extremos locales}
Sea $f$ continua en $(x_0 - \delta, x_0 + \delta) \ = \ E_{(x_0,\delta)}$, derivable en %
$(x_0 - \delta, x_0 + \delta) - \{x_0\} \ = \ E^{*}_{(x_0,\delta)}$:
\begin{gather*}
	1) \quad f'(x) > 0 \ \ \text{en} \ \ (x_0 - \lambda, x_0), \ \ \text{y} \ \ f'(x) < 0 \ \text{en} \ \ (x_0, x_0 + \lambda) \\
	\Rightarrow \ x_0 \ \text{es máximo local} \\ \\
	2) \quad f'(x) < 0 \ \ \text{en} \ \ (x_0 - \lambda, x_0), \ \ \text{y} \ \ f'(x) > 0 \ \text{en} \ \ (x_0, x_0 + \lambda) \\
	\Rightarrow \ x_0 \ \text{es mínimo local}
\end{gather*}
\begin{center}
	\footnotesize(nunca está de más hacer la recta con las flechas de creciendo y de decreciendo, representando %
	el signo de la derivada a ambos lados del $x_0$)
\end{center}
	 
\subsection{Concavidad}
\littleTitle{Teorema} \par
Sea $f: (a,b) \to \mathbb{R}$ dos veces derivable:
\begin{gather*}
	1) \quad f''(x) > 0 \quad \forall x \in (a,b,) \ \Rightarrow \ gr(f) \ \text{es cóncavo positivo; es $\cup$}. \\ \\ 
	2) \quad f''(x) < 0 \quad \forall x \in (a,b,) \ \Rightarrow \ gr(f) \ \text{es cóncavo negativo; es $\cap$}
\end{gather*}
\begin{center}
	\footnotesize (de ser necesario, usar \hyperref[ctb]{\emph{C.T.B.}} para asegurar signos de $f''$)
\end{center}

\subsection{Puntos de inflexión - \emph{P.Inf.}}
$(x_0, f(x_0))$ es \emph{P.Inf.} del $gr(f)$ si
\begin{itemize}
	\item $f$ es continua en $x_0 \in Dom(f)$
	\item $gr(f)$ tiene \emph{concavidad distinta} a ambos lados del probable \emph{P.Inf.}
\end{itemize}
\littleTitle{Observación} \par
Por como está definido el \emph{P.Inf.}, puede ocurrir que $\nexists f''(x_0)$ y/o $\nexists f'(x_0)$

\subsection{Criterio de la 2da derivada para extremos}
Sea $f$ dos veces derivable en $x_0$:
\begin{gather*}
	1) \quad f'(x_0) = 0 \quad \land \quad \overbrace{f''(x_0) > 0}^{c\acute{o}nc. +} \qquad \Rightarrow \quad m %
	\ \text{local en} \ x_0 \\ \\
	2) \quad f'(x_0) = 0 \quad \land \quad \underbrace{f''(x_0) < 0}_{c\acute{o}nc. -} \qquad \Rightarrow \quad M %
	\ \text{local en} \ x_0
\end{gather*}
\begin{center}
	\footnotesize{(regla que no aplica si $f''(x_0) = 0 \quad \lor \quad \nexists f''(x_0)$; en estos casos, %
	puede o no $\exists$ extremo)}
\end{center}

\subsection{Optimización de funciones}
\subsubsection{Conceptos}
\begin{itemize}
	\item \underline{$S$, supremo:} toda la $Im(f)$ es $\leq$ al $S$; $S \in cod{f}, \ S \in \mathbb{R}, \ S \neq \infty$.
	\item \underline{$M$, máximo:} si $S \in Im(f) \ \Rightarrow \ \text{es} \ M$.
	\item \underline{$i$, ínfimo:} toda la $Im(f)$ es $geq$ al $i$; $i \in cod(f), \ i \in \mathbb{R}, \ i \neq \infty$.
	\item \underline{$m$, mínimo:} si $i \in Im(f) \ \Rightarrow \ \text{es} \ m$.
\end{itemize}

\subsubsection{Análisis en funciones definidas en un intervalo cerrado}
\begin{center}
	$f: [a,b] = I \to R$
\end{center} \par
Por \hyperref[tw]{\emph{Weierstrass}}, ya se sabe que $\exists m$ y $\exists M$.
\begin{itemize}
	\item[1°] Los extremos de $f$ están en $a \ \lor \ b \ \lor \in (a,b)$
	\item[2°] Si están en $(a,b)$, el extremo ocurre en un \emph{PC}.
	\item[3°] Comparar $f(a)$ con $f(b)$ con $f(x_0)$, siendo $x_0$ cada \emph{PC}. El mayor valor será $M$, %
	el menor será $m$.
\end{itemize}

\subsubsection{Análisis en funciones definidas en un intervalo genérico}
\begin{center}
	$f: I \to R$, $f$ continua, $I \subseteq \mathbb{R}$, $I$ un intervalo
\end{center} \par
\littleTitle{Premisas} \par
\begin{itemize}
	\item $a < b$ son extremos de $I$, $a \in \mathbb{R}$ o $a = -\infty$, $b \in \mathbb{R}$ o $b = +\infty$
	\item $a,b \in I$ o $a,b \notin I$ (o las otras dos posibilidades)
	\item $\commLim{x}{a^+} f(x) = L_a, \ L_a \in \mathbb{R} \quad \lor \quad L_a = \pm \infty$
	\item $\commLim{x}{b^-} f(x) = L_b, \ L_b \in \mathbb{R} \quad \lor \quad L_b = \pm \infty$
\end{itemize}
Entonces, se tiene 2 casos:
\begin{itemize}
	\item[1°] $\quad L_a = \ \stackrel{-\infty}{+\infty} \quad \lor \quad L_b = \ \stackrel{-\infty}{+\infty} \qquad \Rightarrow \quad \stackrel{\nexists i}{\nexists S} \quad \Rightarrow \quad \stackrel{\nexists m}{\nexists M}$
	\item[2°] $\quad L_a \neq \ \stackrel{-\infty}{+\infty} \quad \land \quad L_b \neq \ \stackrel{-\infty}{+\infty}$, comparar $L_a$ con $L_b$ con $f$ evaluada en cada \emph{PC}. El $\stackrel{\text{menor}}{\text{mayor}}$ será el $\stackrel{i}{S}$. Si $\stackrel{i}{S} \ \in Im(f) \ \Rightarrow \ $ es $\stackrel{m}{M}$. 
\end{itemize}

\vspace{1cm}
\section{Polinomios de Taylor}


\vspace{1cm}
\section{Integral Indefinida - Primitivas}


\vspace{1cm}
\section{Integral Definida}


\vspace{1cm}
\section{Aplicaciones de la Integral}

\ifbool{DEBUG}{
}{
	\newpage
	\appendix
	\begin{appendices}
		\includepdf[pages={1},pagecommand=\section{Trigonometría 1}]{Trig_Cheat_Sheet.pdf}
		\includepdf[pages={2,3,4}]{Trig_Cheat_Sheet.pdf}
		\includepdf[pages={1},pagecommand=\section{Trigonometría 2}]{FormulaSheet.pdf}
	\end{appendices}
}




\end{document}

